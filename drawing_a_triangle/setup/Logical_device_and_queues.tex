\section{逻辑设备和队列}

\subsection{介绍}

选择物理设备后,我们还需要一个逻辑设备来作为和物理设备交互的接口。逻辑设备的创建过程类似于我们之前描述的Vulkan实例的创建过程。我们还需要指定使用的队列所属的队列族。对于同一个物理设备,我们可以根据需求的不同,创建多个逻辑设备。

首先,我们添加一个逻辑设备对象作为类成员:

\begin{lstlisting}[language={[ANSI]C}]
VkDevice device;
\end{lstlisting}

接着,添加一个叫做createLogicalDevice的函数,在initVulkan函数中调用它。

\begin{lstlisting}[language={[ANSI]C}]
void initVulkan() {
	createInstance();
	setupDebugCallback();
	pickPhysicalDevice();
	createLogicalDevice();
}

void createLogicalDevice() {

}
\end{lstlisting}

\subsection{指定要创建的队列}

逻辑设备创建需要填写VkDeviceQueueCreateInfo结构体。这一结构体描述了针对一个队列族我们所需的队列数量。目前而言,我们只使用了带有图形能力的队列族。

\begin{lstlisting}[language={[ANSI]C}]
QueueFamilyIndices indices = findQueueFamilies(physicalDevice);

VkDeviceQueueCreateInfo queueCreateInfo = {};
queueCreateInfo.sType = VK_STRUCTURE_TYPE_DEVICE_QUEUE_CREATE_INFO;
queueCreateInfo.queueFamilyIndex = indices.graphicsFamily;
queueCreateInfo.queueCount = 1;
\end{lstlisting}

目前而言,对于每个队列族,驱动程序只允许创建很少数量的队列,但实际上,对于每一个队列族,我们很少需要一个以上的队列。我们可以在多个线程创建指令缓冲,然后在主线程一次将它们全部提交,降低调用开销。

Vulkan需要我们赋予队列一个0.0到1.0之间的浮点数作为优先级来控制指令缓冲的执行顺序。即使只有一个队列,我们也要显式地赋予队列优先级:

\begin{lstlisting}[language={[ANSI]C}]
float queuePriority = 1.0f;
queueCreateInfo.pQueuePriorities = &queuePriority;
\end{lstlisting}

\subsection{指定使用的设备特性}

接下来,我们要指定应用程序使用的设备特性。我们暂时先简单地定义它,之后再回来填写:

\begin{lstlisting}[language={[ANSI]C}]
VkPhysicalDeviceFeatures deviceFeatures = {};
\end{lstlisting}

\subsection{创建逻辑设备}

填写好前面两个结构体后,我们可以开始填写VkDeviceCreateInfo结构体。

\begin{lstlisting}[language={[ANSI]C}]
VkDeviceCreateInfo createInfo = {};
createInfo.sType = VK_STRUCTURE_TYPE_DEVICE_CREATE_INFO;
\end{lstlisting}

将VkDeviceCreateInfo结构体的pQueueCreateInfos指针指向queueCreateInfo的地址,pEnabledFeatures指针指向deviceFeatures的地址:

\begin{lstlisting}[language={[ANSI]C}]
createInfo.pQueueCreateInfos = &queueCreateInfo;
createInfo.queueCreateInfoCount = 1;

createInfo.pEnabledFeatures = &deviceFeatures;
\end{lstlisting}

结构体的其余成员和VkInstanceCreateInfo类似,不同的是这次的设置是针对设备的。

VK\_KHR\_swapchain就是一个设备特定扩展的例子,这一扩展使得我们可以将渲染的图像在窗口上显示出来。看起来似乎应该所有支持Vulkan的设备都应该支持这一扩展,然而,实际上有的Vulkan设备只支持计算指令,不支持这一图形相关扩展。在之后的章节,我们会对交换链进行更加深入地说明。

之前提到,我们可以对设备和Vulkan实例使用相同地校验层,不需要额外的扩展支持:

\begin{lstlisting}[language={[ANSI]C}]
createInfo.enabledExtensionCount = 0;

if (enableValidationLayers) {
	createInfo.enabledLayerCount = static_cast<uint32_t>(validationLayers.size());
	createInfo.ppEnabledLayerNames = validationLayers.data();
} else {
	createInfo.enabledLayerCount = 0;
}
\end{lstlisting}

现在,我们可以调用vkCreateDevice函数创建逻辑设备了。

\begin{lstlisting}[language={[ANSI]C}]
if (vkCreateDevice(physicalDevice, &createInfo, nullptr, &device) != VK_SUCCESS) {
	throw std::runtime_error("failed to create logical device!");
}
\end{lstlisting}

vkCreateDevice函数的参数包括我们创建的逻辑设备进行交互的物理设备对象,我们刚刚在结构体中指定的需要使用的队列信息,可选的分配器回调,以及用来存储返回的逻辑设备对象的内存地址。和Vulkan实例对象的创建函数类似,这一函数调用在请求的设备需求不被满足时返回错误代码。

逻辑设备对象创建后,应用程序结束前,需要我们自己在cleanup函数中调用vkDestroyDevice函数来清除它:

\begin{lstlisting}[language={[ANSI]C}]
void cleanup() {
	vkDestroyDevice(device, nullptr);
		...
}
\end{lstlisting}

逻辑设备并不直接与Vulkan实例交互,所以创建逻辑设备时不需要使用Vulkan实例作为参数。

\subsection{获取队列句柄}

创建逻辑设备时指定的队列会随着逻辑设备一同被创建,为了方便,我们添加了一个VkQueue成员变量来直接存储逻辑设备的队列句柄:

\begin{lstlisting}[language={[ANSI]C}]
VkQueue graphicsQueue;
\end{lstlisting}

逻辑设备的队列会在逻辑设备清除时,自动被清除,所以不需要我们在cleanup函数中进行队列的清除操作。

vkGetDeviceQueue函数可以获取指定队列族的队列句柄。它的参数依次是逻辑设备对象,队列族索引,队列索引,用来存储返回的队列句柄的内存地址。因为,我们只创建了一个队列,所以,可以直接使用索引0调用函数:

\begin{lstlisting}[language={[ANSI]C}]
vkGetDeviceQueue(device, indices.graphicsFamily, 0, &graphicsQueue);
\end{lstlisting}

创建完逻辑设备,我们就可以真正开始使用显卡来完成一些操作。在接下来的章节,我们将开始配置资源,进行一些绘制操作,将渲染结果显示在窗口上。

本章节代码:

C++:

\url{https://vulkan-tutorial.com/code/04_logical_device.cpp}

\newpage
\