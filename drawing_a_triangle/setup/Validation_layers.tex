\section{校验层}

\subsection{校验层是什么?}

Vulkan API的设计是紧紧围绕最小化驱动程序开销进行的,所以,默认情况下,Vulkan API提供的错误检查功能非常有限。很多很基本的错误都没有被Vulkan显式地处理,遇到错误程序会直接崩溃或者发生未被明确定义的行为。Vukan需要我们显式地定义每一个操作,所以就很容易在使用过程中产生一些小错误,比如使用了一个新的GPU特性,却忘记在逻辑设备创建时请求这一特性。

然而,这并不意味着我们不能将错误检查加入API调用。Vulkan引入了校验层来优雅地解决这个问题。校验层是一个可选的可以用来在Vulkan API函数调用上进行附加操作的组件。校验层常被用来做下面的工作:

\begin{itemize}
	\item 检测参数值是否合法
	\item 追踪对象的创建和清除操作,发现资源泄漏问题
	\item 追踪调用来自的线程,检测是否线程安全。
	\item 将API调用和调用的参数写入日志
	\item 追踪API调用进行分析和回放
\end{itemize}

下面的代码演示了Vulkan的校验层是如何工作的:

\begin{lstlisting}[language={[ANSI]C}]
VkResult vkCreateInstance(
const VkInstanceCreateInfo* pCreateInfo,
const VkAllocationCallbacks* pAllocator,
VkInstance* instance) {

	if (pCreateInfo == nullptr || instance == nullptr) {
		log("Null pointer passed to required parameter!");
		return VK_ERROR_INITIALIZATION_FAILED;
	}

	return real_vkCreateInstance(pCreateInfo, pAllocator, instance);
}
\end{lstlisting}

校验层可以被自由堆叠包含任何读者感兴趣的调试功能。我们可以在开发时使用校验层,然后在发布应用程序时,禁用校验层来提高程序的运行表现。

Vulkan库本身并没有提供任何内建的校验层,但LunarG的Vulkan SDK提供了一个非常不错的校验层实现。读者可以使用这个校验层实现来保证自己的应用程序在不同的驱动程序下能够尽可能得表现一致,而不是依赖于某个驱动程序的未定义行为。

校验层只能用于安装了它们的系统,比如,LunarG的校验层只可以在安装了Vulkan SDK的PC上使用。

Vulkan可以使用两种不同类型的校验层:实例校验层和设备校验层。实例校验层只检查和全局Vulkan对象相关的调用,比如Vulkan实例。设备校验层只检查和特定GPU相关的调用。设备校验层现在已经不推荐使用,也就是说,应该使用实例校验层来检测所有的Vulkan调用。Vulkan规范文档为了兼容性仍推荐启用设备校验层。在本教程,为了简便,我们为实例和设备指定相同的校验层。

\subsection{使用校验层}

在本章节,我们将使用LunarG的Vulkan SDK提供的校验层。和使用扩展一样,使用校验层需要指定校验层的名称。LunarG的Vulkan SDK允许我们通过VK\_LAYER\_KHRONOS\_validation来隐式地开启所有可用的校验层。

首先,让我们添加两个变量到程序中来控制是否启用指定的校验层。这里,我们通过条件编译来设定是否启用校验层。代码中的NDEBUG宏是C++标准的一部分,表示是否处于非调试模式下:

\begin{lstlisting}[language={[ANSI]C}]
const int WIDTH = 800;
const int HEIGHT = 600;

const std::vector<const char*> validationLayers = {
	"VK_LAYER_KHRONOS_validation"
};

#ifdef NDEBUG
const bool enableValidationLayers = false;
#else
const bool enableValidationLayers = true;
#endif
\end{lstlisting}

接着,我们添加了一个叫做checkValidationLayerSupport的函数来请求所有可用的校验层。首先,我们调用vkEnumerateInstanceLayerProperties函数获取了所有可用的校验层列表。这一函数的用法和前面我们在创建Vulkan实例章节中使用的vkEnumerateInstanceExtensionProperties函数相同。

\begin{lstlisting}[language={[ANSI]C}]
bool checkValidationLayerSupport() {
	uint32_t layerCount;
	vkEnumerateInstanceLayerProperties(&layerCount, nullptr);

	std::vector<VkLayerProperties> availableLayers(layerCount);
	vkEnumerateInstanceLayerProperties(&layerCount, availableLayers.data());

	return false;
}
\end{lstlisting}

接着,检查是否所有validationLayers列表中的校验层都可以在availableLayers列表中找到:

\begin{lstlisting}[language={[ANSI]C}]
for (const char* layerName : validationLayers) {
	bool layerFound = false;

	for (const auto& layerProperties : availableLayers) {
		if (strcmp(layerName, layerProperties.layerName) == 0) {
			layerFound = true;
			break;
		}
	}

	if (!layerFound) {
		return false;
	}
}
return true;
\end{lstlisting}

现在,我们在createInstance函数中调用它:

\begin{lstlisting}[language={[ANSI]C}]
void createInstance() {
	if (enableValidationLayers && !checkValidationLayerSupport()) {
		throw std::runtime_error("validation layers requested, but not available!");
	}

		...
}
\end{lstlisting}

现在,在调试模式下编译运行程序,确保没有错误出现。如果程序运行时出现错误,请确保正确安装了Vulkan SDK。如果程序报告缺少可用的校验层,可以查阅LunarG的Vulkan SDK的官方文档寻找解决方法。

最后,修改我们之前的填写的VkInstanceCreateInfo结构体信息,在校验层启用时使用校验层:

\begin{lstlisting}[language={[ANSI]C}]
if (enableValidationLayers) {
	createInfo.enabledLayerCount = static_cast<uint32_t>(validationLayers.size());
	createInfo.ppEnabledLayerNames = validationLayers.data();
} else {
	createInfo.enabledLayerCount = 0;
}
\end{lstlisting}

如果校验层检查成功,vkCreateInstance函数调用就不会返回VK\_ERROR\_LAYER\_NOT\_PRESENT这一错误代码,但为了保险起见,读者应该运行程序来确保没有问题出现。

\subsection{消息回调}

仅仅启用校验层并没有任何用处,我们不能得到任何有用的调试信息。为了获得调试信息,我们需要使用VK\_EXT\_debug\_utils扩展,设置回调函数来接受调试信息。

我们添加了一个叫做getRequiredExtensions的函数,这一函数根据是否启用校验层,返回所需的扩展列表:

\begin{lstlisting}[language={[ANSI]C}]
std::vector<const char*> getRequiredExtensions() {
	uint32_t glfwExtensionCount = 0;
	const char** glfwExtensions;
	glfwExtensions = glfwGetRequiredInstanceExtensions(&glfwExtensionCount);

	std::vector<const char*> extensions(glfwExtensions,
	glfwExtensions + glfwExtensionCount);

	if (enableValidationLayers) {
		extensions.push_back(VK_EXT_DEBUG_UTILS_EXTENSION_NAME);
	}

	return extensions;
}
\end{lstlisting}

GLFW指定的扩展是必需的,调试报告相关的扩展根据校验层是否启用添加。代码中我们使用了一个VK\_EXT\_DEBUG\_UTILS\_EXTENSION\_NAME,它等价于VK\_EXT\_debug\_utils,使用它是为了避免打字时的手误。

现在,我们在createInstance函数中调用这一函数:

\begin{lstlisting}[language={[ANSI]C}]
auto extensions = getRequiredExtensions();
createInfo.enabledExtensionCount = static_cast<uint32_t>(extensions.size());
createInfo.ppEnabledExtensionNames = extensions.data();
\end{lstlisting}

接着,编译运行程序,确保没有出现VK\_ERROR\_EXTENSION\_NOT\_PRESENT错误。校验层的可用已经隐含说明对应的扩展存在,所以我们不需要额外去做扩展是否存在的检查。

现在,让我们来看接受调试信息的回调函数。我们在程序中以vkDebugUtilsMessengerCallbackEXT为原型添加一个叫做debugCallback的静态函数。这一函数使用VKAPI\_ATTR和VKAPI\_CALL定义,确保它可以被Vulkan库调用。

\begin{lstlisting}[language={[ANSI]C}]
static VKAPI_ATTR VkBool32 VKAPI_CALL debugCallback( VkDebugUtilsMessageSeverityFlagBitsEXT messageSeverity, VkDebugUtilsMessageTypeFlagsEXT messageType, const VkDebugUtilsMessengerCallbackDataEXT* pCallbackData, void* pUserData) {

	std::cerr << "validation layer: " << pCallbackData->pMessage << std::endl;

	return VK_FALSE;
}
\end{lstlisting}

函数的第一个参数指定了消息的级别,它可以是下面这些值:

\begin{itemize}
	\item VK\_DEBUG\_UTILS\_MESSAGE\_SEVERITY\_VERBOSE\_BIT\_EXT:诊断信息
	\item VK\_DEBUG\_UTILS\_MESSAGE\_SEVERITY\_INFO\_BIT\_EXT:资源创建之类的信息
	\item VK\_DEBUG\_UTILS\_MESSAGE\_SEVERITY\_WARNING\_BIT\_EXT:警告信息
	\item VK\_DEBUG\_UTILS\_MESSAGE\_SEVERITY\_ERROR\_BIT\_EXT:不合法和可能造成崩溃的操作信息
\end{itemize}

这些值经过一定设计,可以使用比较运算符来过滤处理一定级别以上的调试信息:

\begin{lstlisting}[language={[ANSI]C}]
if (messageSeverity >=VK_DEBUG_UTILS_MESSAGE_SEVERITY_WARNING_BIT_EXT) {
	// Message is important enough to show
}
\end{lstlisting}

messageType参数可以是下面这些值:

\begin{itemize}
	\item VK\_DEBUG\_UTILS\_MESSAGE\_TYPE\_GENERAL\_BIT\_EXT:发生了一些与规范和性能无关的事件
	\item VK\_DEBUG\_UTILS\_MESSAGE\_TYPE\_VALIDATION\_BIT\_EXT:出现了违反规范的情况或发生了一个可能的错误
	\item VK\_DEBUG\_UTILS\_MESSAGE\_TYPE\_PERFORMANCE\_BIT\_EXT:进行了可能影响Vulkan性能的行为
\end{itemize}

pCallbackData参数是一个指向VkDebugUtilsMessengerCallbackDataEXT结构体的指针,这一结构体包含了下面这些非常重要的成员:

\begin{itemize}
	\item pMessage:一个以null结尾的包含调试信息的字符串
	\item pObjects:存储有和消息相关的Vulkan对象句柄的数组
	\item objectCount:数组中的对象个数
\end{itemize}

最后一个参数pUserData是一个指向了我们设置回调函数时,传递的数据的指针。

回调函数返回了一个布尔值,用来表示引发校验层处理的Vulkan API调用是否被中断。如果返回值为true,对应Vulkan API调用就会返回VK\_ERROR\_VALIDATION\_FAILED\_EXT错误代码。通常,只在测试校验层本身时会返回true,其余情况下,回调函数应该返回VK\_FALSE。

定义完回调函数,接下来要做的就是设置Vulkan使用这一回调函数。我们需要一个VkDebugUtilsMessengerEXT对象来存储回调函数信息,然后将它提交给Vulkan完成回调函数的设置:

\begin{lstlisting}[language={[ANSI]C}]
VkDebugUtilsMessengerEXT callback;
\end{lstlisting}

现在,我们在initVulkan函数中,在createInstance函数调用之后添加一个setupDebugCallback函数调用:

\begin{lstlisting}[language={[ANSI]C}]
void initVulkan() {
	createInstance();
	setupDebugCallback();
}

void setupDebugCallback() {
	if (!enableValidationLayers) return;

}
\end{lstlisting}

接着,我们需要填写VkDebugUtilsMessengerCreateInfoEXT结构体所需的信息:

\begin{lstlisting}[language={[ANSI]C}]
VkDebugUtilsMessengerCreateInfoEXT createInfo = {};
createInfo.sType = VK_STRUCTURE_TYPE_DEBUG_UTILS_MESSENGER_CREATE_INFO_EXT;
createInfo.messageSeverity = VK_DEBUG_UTILS_MESSAGE_SEVERITY_VERBOSE_BIT_EXT |
	VK_DEBUG_UTILS_MESSAGE_SEVERITY_WARNING_BIT_EXT |
	VK_DEBUG_UTILS_MESSAGE_SEVERITY_ERROR_BIT_EXT;
createInfo.messageType = VK_DEBUG_UTILS_MESSAGE_TYPE_GENERAL_BIT_EXT |
	VK_DEBUG_UTILS_MESSAGE_TYPE_VALIDATION_BIT_EXT |
	VK_DEBUG_UTILS_MESSAGE_TYPE_PERFORMANCE_BIT_EXT;
createInfo.pfnUserCallback = debugCallback;
createInfo.pUserData = nullptr; // Optional
\end{lstlisting}

messageSeverity域可以用来指定回调函数处理的消息级别。在这里,我们设置回调函数处理除了VK\_DEBUG\_UTILS\_MESSAGE\_SEVERITY\_INFO\_BIT\_EXT外的所有级别的消息,这使得我们的回调函数可以接收到可能的问题信息,同时忽略掉冗长的一般调试信息。

messageType域用来指定回调函数处理的消息类型。在这里,我们设置处理所有类型的消息。读者可以根据自己的需要开启和禁用处理的消息类型。

pfnUserCallback域是一个指向回调函数的指针。pUserData是一个指向用户自定义数据的指针,它是可选的,这个指针所指的地址会被作为回调函数的参数,用来向回调函数传递用户数据。

有许多方式配置校验层消息和回调,更多信息可以参考扩展的规范文档。

填写完结构体信息后,我们将它作为一个参数调用vkCreateDebugUtilsMessengerEXT函数来创建VkDebugUtilsMessengerEXT对象。由于vkCreateDebugUtilsMessengerEXT函数是一个扩展函数,不会被Vulkan库自动加载,所以需要我们自己使用vkGetInstanceProcAddr函数来加载它。在这里,我们创建了一个代理函数,来载入vkCreateDebugUtilsMessengerEXT函数:

\begin{lstlisting}[language={[ANSI]C}]
VkResult CreateDebugUtilsMessengerEXT(VkInstance instance, const VkDebugUtilsMessengerCreateInfoEXT* pCreateInfo, const VkAllocationCallbacks* pAllocator, VkDebugUtilsMessengerEXT* pCallback) {
	auto func = (PFN_vkCreateDebugUtilsMessengerEXT)
	vkGetInstanceProcAddr(instance, "vkCreateDebugUtilsMessengerEXT");
	if (func != nullptr) {
		return func(instance, pCreateInfo, pAllocator, pCallback);
	} else {
		return VK_ERROR_EXTENSION_NOT_PRESENT;
	}
}
\end{lstlisting}

vkGetInstanceProcAddr函数如果不能被加载,那么我们的代理函数就会发挥nullptr。现在我们可以使用这个代理函数来创建扩展对象:

\begin{lstlisting}[language={[ANSI]C}]
if (CreateDebugUtilsMessengerEXT(instance, &createInfo, nullptr,
&callback) != VK_SUCCESS) {
	throw std::runtime_error("failed to set up debug callback!");
}
\end{lstlisting}

函数的第二个参数是可选的分配器回调函数,我们没有自定义的分配器,所以将其设置为nullptr。由于我们的调试回调是针对特定Vulkan实例和它的校验层,所以需要在第一个参数指定调试回调作用的Vulkan实例。现在,让我们编译运行程序,如果一切顺利,读者可以看到一个空白窗口,关闭空白窗口后,可以在控制台窗口看到下面的信息:

\begin{figure}[H]
	\centering
	\includegraphics[scale=0.55]{img/f6-1.jpg}
\end{figure}

这说明,我们的程序还存在问题!VkDebugUtilsMessengerEXT对象在程序结束前通过调用vkDestroyDebugUtilsMessengerEXT函数来清除掉。和vkCreateDebugUtilsMessengerEXT函数相同,Vulkan库没有自动加载这个函数,需要我们自己加载它。控制台窗口出现多次相同的错误信息是正常的,这是因为有多个校验层检查发现了这个问题。

现在,让我们创建CreateDebugUtilsMessengerEXT函数的代理函数:

\begin{lstlisting}[language={[ANSI]C}]
void DestroyDebugUtilsMessengerEXT(VkInstance instance, VkDebugUtilsMessengerEXT callback, const VkAllocationCallbacks* pAllocator) {
	auto func = (PFN_vkDestroyDebugUtilsMessengerEXT) vkGetInstanceProcAddr(instance, "vkDestroyDebugUtilsMessengerEXT");
	if (func != nullptr) {
		func(instance, callback, pAllocator);
	}
}
\end{lstlisting}

这个代理函数需要被定义为类的静态成员函数或者被定义在类之外。我们在cleanup函数中调用这个函数:

\begin{lstlisting}[language={[ANSI]C}]
void cleanup() {
	if (enableValidationLayers) {
		DestroyDebugUtilsMessengerEXT(instance, callback, nullptr);
	}

	vkDestroyInstance(instance, nullptr);

	glfwDestroyWindow(window);

	glfwTerminate();
}
\end{lstlisting}

现在,再次编译运行程序,如果一切顺利,错误信息这次就不会出现。如果读者想要了解到底是哪个函数调用引发了错误消息,可以在处理消息的回调函数设置断点,然后运行程序,观察程序在断点位置时的调用栈,就可以确定引发错误消息的函数调用。

\subsection{配置}

校验层除了VkDebugUtilsMessengerCreateInfoEXT结构体指定的标志外,还有大量可以决定校验层行为的设置。读者可以浏览Vulkan SDK的Config目录,里面有一个vk\_layer\_settings.txt解释了如何配置校验层。

读者可以将vk\_layer\_settings.txt 复制到自己的项目的Debug和Release目录来使用它,并按照说明根据需要修改设置。在本教程,我们只使用vk\_layer\_settings.txt的默认设置。

在之后的章节,我们会故意造成一些错误,来演示如何使用校验层来发现这些错误,帮助读者理解校验层的重要性。现在,是时候来看一看系统中的Vulkan设备了。

本章节代码:

C++:

\url{https://vulkan-tutorial.com/code/02_validation_layers.cpp}

\newpage
