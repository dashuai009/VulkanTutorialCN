\section{实例}

\subsection{创建一个实例}

我们首先创建一个实例来初始化Vulkan库。这个实例指定了一些驱动程序需要使用的应用程序信息。

我们添加了一个createInstance函数调用到initVulkan函数中:

\begin{lstlisting}[language={[ANSI]C}]
void initVulkan() {
	createInstance();
}
\end{lstlisting}

添加了一个存储实例句柄的私有成员:

\begin{lstlisting}[language={[ANSI]C}]
private:
	VkInstance instance;
\end{lstlisting}

然后,填写应用程序信息,这些信息的填写不是必须的,但填写的信息可能会作为驱动程序的优化依据,让驱动程序进行一些特殊的优化。比如,应用程序使用了某个引擎,驱动程序对这个引擎有一些特殊处理,这时就可能有很大的优化提升:

\begin{lstlisting}[language={[ANSI]C}]
VkApplicationInfo appInfo = {};
appInfo.sType = VK_STRUCTURE_TYPE_APPLICATION_INFO;
appInfo.pApplicationName = "Hello Triangle";
appInfo.applicationVersion = VK_MAKE_VERSION(1, 0, 0);
appInfo.pEngineName = "No Engine";
appInfo.engineVersion = VK_MAKE_VERSION(1, 0, 0);
appInfo.apiVersion = VK_API_VERSION_1_0;
\end{lstlisting}

之前提到,Vulkan的很多结构体需要我们显式地在sType成员变量中指定结构体的类型。此外,许多Vulkan的结构体还有一个pNext成员变量,用来指向未来可能扩展的参数信息,现在,我们并没有使用它,将其设置为nullptr。

Vulkan倾向于通过结构体传递信息,我们需要填写一个或多个结构体来提供足够的信息创建Vulkan实例。下面的这个结构体是必须的,它告诉Vulkan的驱动程序需要使用的全局扩展和校验层。全局是指这里的设置对于整个应用程序都有效,而不仅仅对一个设备有效,在之后的章节,我们会对此有更加清晰得认识。

\begin{lstlisting}[language={[ANSI]C}]
VkInstanceCreateInfo createInfo = {};
createInfo.sType = VK_STRUCTURE_TYPE_INSTANCE_CREATE_INFO;
createInfo.pApplicationInfo = &appInfo;
\end{lstlisting}

上面代码中填写得两个参数非常直白,不用多解释。接下来,我们需要指定需要的全局扩展。之前提到,Vulkan是平台无关的API,所以需要一个和窗口系统交互的扩展。GLFW库包含了一个可以返回这一扩展的函数,我们可以直接使用它:

\begin{lstlisting}[language={[ANSI]C}]
uint32_t glfwExtensionCount = 0;
const char** glfwExtensions;

glfwExtensions = glfwGetRequiredInstanceExtensions(&glfwExtensionCount);

createInfo.enabledExtensionCount = glfwExtensionCount;
createInfo.ppEnabledExtensionNames = glfwExtensions;
\end{lstlisting}

结构体的最后两个成员变量用来指定全局校验层。我们将在之后的章节更加深入地讨论它,在这里,我们将其设置为0,不使用它:

\begin{lstlisting}[language={[ANSI]C}]
createInfo.enabledLayerCount = 0;
\end{lstlisting}

填写完所有必要的信息,我们就可以调用vkCreateInstance函数来创建Vulkan实例:

\begin{lstlisting}[language={[ANSI]C}]
VkResult result = vkCreateInstance(&createInfo, nullptr, &instance);
\end{lstlisting}

如你所看到的,创建Vulkan对象的函数参数的一般形式就是:

\begin{itemize}
	\item 一个包含了创建信息的结构体指针
	\item 一个自定义的分配器回调函数,在本教程,我们没有使用自定义的分配器,总是将它设置为nullptr
	\item 一个指向新对象句柄存储位置的指针
\end{itemize}

如果一切顺利,我们创建的实例的句柄就被存储在了类的VkInstance成员变量中。几乎所有Vulkan API函数调用都会返回一个VkResult来反应函数调用的结果,它的值可以是VK\_SUCESS表示调用成功,或是一个错误代码,表示调用失败。为了检测实例是否创建成功,我们可以直接将创建函数在条件语句中使用,不需要存储它的返回值:

\begin{lstlisting}[language={[ANSI]C}]
if (vkCreateInstance(&createInfo, nullptr, &instance) != VK_SUCCESS) {
	throw std::runtime_error("failed to create instance!");
}
\end{lstlisting}

现在可以编译运行程序来确保实例创建成功。

\subsection{检测扩展支持}

如果读者看过vkCreateInstance函数的官方文档,可能会知道它返回的之中一个错误代码VK\_ERROR\_EXTENSION\_NOT\_PRESENT。我们可以利用这个错误代码在扩展不能满足时直接结束我们的程序,这对于像窗口系统这种必要的扩展来说非常适合。但有时,我们请求的扩展可能是非必须的,有了很好,没有的话,程序仍然可以运行,这时,我们该怎么做呢?

实际上Vulkan提供了一个叫做vkEnumerateInstanceExtensionProperties可以在Vulkan实例创建之前返回支持的扩展列表。通过它,我们可以获取扩展的个数,以及扩展的详细信息,此外,它还允许我们指定校验层来对扩展进行过滤,但在这里,我们不使用它,将其设置为nullptr。

我们首先需要知道扩展的数量,以便分配合适的数组大小来存储信息。可以通过下面的代码来获取扩展的数量:

\begin{lstlisting}[language={[ANSI]C}]
uint32_t extensionCount = 0;
vkEnumerateInstanceExtensionProperties(nullptr, &extensionCount, nullptr);
\end{lstlisting}

知道了扩展的数量后,我们就可以分配数组来存储扩展信息:

\begin{lstlisting}[language={[ANSI]C}]
std::vector<VkExtensionProperties> extensions(extensionCount);
\end{lstlisting}

我们使用下面的代码获取所有扩展信息:

\begin{lstlisting}[language={[ANSI]C}]
vkEnumerateInstanceExtensionProperties(nullptr, &extensionCount, extensions.data());
\end{lstlisting}

每个VkExtensionProperties结构体包含了扩展的名字和版本信息。我们可以使用下面的代码将这些信息打印在控制台窗口中(代码中的\\t表示制表符):

\begin{lstlisting}[language={[ANSI]C}]
std::cout << "available extensions:" << std::endl;

for (const auto& extension : extensions) {
	std::cout << "\t" << extension.extensionName << std::endl;
}
\end{lstlisting}

读者可以将上面的代码加入createInstance函数,获取扩展支持信息。此外,我们可以编写一个函数来检测调用glfwGetRequiredInstanceExtensions函数返回的扩展是否全部包含在了扩展支持列表中。

\subsection{清理}

VkInstance应该在应用程序结束前进行清除操作。我们可以在cleanup中调用vkDestroyInstance函数完成清除工作:

\begin{lstlisting}[language={[ANSI]C}]
void cleanup() {
	vkDestroyInstance(instance, nullptr);

	glfwDestroyWindow(window);

	glfwTerminate();
}
\end{lstlisting}

vkDestroyInstance函数的参数非常直白。之前提到,Vulkan对象的分配和清除函数都有一个可选的分配器回调参数,在本教程,我们没有自定义的分配器,所以,将其设置为nullptr。除了Vulkan实例,其余我们使用Vulkan API创建的对象也需要被清除,且应该在Vulkan实例清除之前被清除。

创建Vulkan实例后,在进行更复杂的操作之前,我们先熟悉一下校验层来帮助我们进行应用程序的调试。

本章节代码:

C++:

\url{https://vulkan-tutorial.com/code/01_instance_creation.cpp}

\newpage
