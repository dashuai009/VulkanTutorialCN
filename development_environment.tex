\section{开发环境}

在本章节,我们将介绍如何配置Vulkan SDK和一些非常有用的库。所有在这里用到的工具除了编译器外,适用于Windows,Linux和MacOS三个平台,但它们的安装方法可能在不同平台会有所不同,所以在这里我们按平台分别描述如何配置它们。

\subsection{Windows}

我们这里使用Visual Studio 2017作为Windows平台的开发环境,当然使用Visual Studio 2013或2015应该也不会有任何问题,只是配置方法可能会略微不同。

\subsubsection{Vulkan SDK}

Vulkan SDK是使用Vulkan开发应用程序必不可少的组件。它包含了Vulkan API的头文件,一个校验层实现,调试工具和Vulkan函数加载器。Vulkan函数加载器类似OpenGL的GLEW可以在运行时查询驱动程序支持的Vulkan API函数。

Vulkan SDK可以从LunarG的网站上免费下载。

\begin{figure}[H]
	\centering
	\includegraphics[scale=0.5]{img/f3-1.jpg}
\end{figure}

安装Vulkan SDK后,我们需要验证下我们的显卡和驱动程序是否支持Vulkan。这可以通过运行Vulkan SDK自带的cube.exe来完成,我们可以在Vulkan SDK安装目录下的Bin目录下找到它,运行后,可以看下到下面的窗口:

\begin{figure}[H]
	\centering
	\includegraphics[scale=0.5]{img/f3-2.jpg}
\end{figure}

如果没有看到这个窗口,而是出现了一条错误消息,可以尝试更新显卡的驱动程序到最新版本,再次尝试,如果仍然出现错误消息,可以在显卡官网查询自己的显卡是否支持Vukan。

在Bin目录下还有一个非常有用的程序:glslangValidator.exe。它可以将GLSL代码编译为字节码。我们会在着色器模块章节,对它进行更为详细地说明。除此之外,Bin目录下还包含了Vulkan函数加载器和校验层的二进制文件,它们的库文件则位于Vulkan SDK的Lib目录下。

Vulkan SDK的Documentation目录包含了Vulkan SDK的离线文档和完整的Vulkan规范文档。最后是Vulkan SDK的Include目录,它包含了Vulkan API的头文件。除此之外,还有很多文件和目录,但对于我们的教程来说,并没有直接用到它们,所以就不再一一介绍。

\subsubsection{GLFW}

之前提到,Vulkan是一个平台无关的图形API,它没有包含任何用于创建窗口的功能。为了跨平台和避免陷入Win32的窗口细节中去,我们使用GLFW库来完成窗口相关操作,GLFW库支持Windows,Linux和MacOS。当然,还有其它一些库可以完成类似功能,比如SDL。但除了窗口相关处理,GLFW库对于Vulkan的使用还有其它一些优点。

读者可以再GLFW的官方网站上免费下载到最新版本的GLFW库。在本教程,我们使用64位版本的GLFW库,但32位版本也是可以的,只是编译使用Vulkan的应用程序时也需要链接到32位版本的Vulkan API,也就是链接到Vulkan SDK下Lib32目录下的库。下载GLFW后,将它解压缩到一个合适的位置。这里,我们将它解压到Visual Studio目录下的Libraries目录中。不要纠结于为什么解压后不存在libvc-2017目录,我们的Visual Studio 2017是完全兼容lib-vc2015的。

\begin{figure}[H]
	\centering
	\includegraphics[scale=0.5]{img/f3-3.png}
\end{figure}

\subsubsection{GLM}

和DirectX 12不同,Vulkan没有包含线性代数库,我们需要自己找一个。GLM就是一个我们需要的线性代数库,它经常和OpenGL一块使用。

GLM是一个只有头文件的库,我们只需要下载它的最新版,然后将它放在一个合适的位置,就可以通过包含头文件的方式使用它。

\begin{figure}[H]
	\centering
	\includegraphics[scale=0.5]{img/f3-4.jpg}
\end{figure}

\subsubsection{配置Visual Studio}

现在,我们可以创建一个基本Visual Studio工程来验证我们安装的依赖是否可以正常工作。

首先,启动Visual Studio,然后选择Windows Destop Wizard。

\begin{figure}[H]
	\centering
	\includegraphics[scale=0.5]{img/f3-5.jpg}
\end{figure}

我们选择使用Console Application (.exe)应用程序类型,这样做我们就可以直接将调试信息输出到控制台窗口上。另外,我们将Empty Project选项打勾来阻止Visual Studo添加模板代码。

\begin{figure}[H]
	\centering
	\includegraphics[scale=0.5]{img/f3-6.jpg}
\end{figure}

创建项目后,我们添加一个C++源代码文件到项目中。

\begin{figure}[H]
	\centering
	\includegraphics[scale=0.5]{img/f3-7.jpg}
\end{figure}

\begin{figure}[H]
	\centering
	\includegraphics[scale=0.5]{img/f3-8.jpg}
\end{figure}

下面的代码是这个C++源文件的内容。源代码的内容暂时不需要理解,我们现在只是为了验证我们的依赖是否配置正确,源代码的内容,我们会在后面的章节详细说明。

\begin{lstlisting}[language={[ANSI]C}]
#define GLFW_INCLUDE_VULKAN
#include <GLFW/glfw3.h>

#define GLM_FORCE_RADIANS
#define GLM_FORCE_DEPTH_ZERO_TO_ONE
#include <glm/vec4.hpp>
#include <glm/mat4x4.hpp>

#include <iostream>

int main() {
	glfwInit();

	glfwWindowHint(GLFW_CLIENT_API, GLFW_NO_API);
	GLFWwindow* window = glfwCreateWindow(800, 600, "Vulkan window", nullptr, nullptr);

	uint32_t extensionCount = 0;
	vkEnumerateInstanceExtensionProperties(nullptr,
				&extensionCount, nullptr);

	std::cout << extensionCount << " extensions supported" << std::endl;

	glm::mat4 matrix;
	glm::vec4 vec;
	auto test = matrix * vec;

	while(!glfwWindowShouldClose(window)) {
		glfwPollEvents();
	}

	glfwDestroyWindow(window);

	glfwTerminate();

	return 0;
}
\end{lstlisting}

现在,让我们开始配置项目属性,选择All Configurations,让设置对Debug和Release模式都有效。

\begin{figure}[H]
	\centering
	\includegraphics[scale=0.5]{img/f3-9.jpg}
\end{figure}

\begin{figure}[H]
	\centering
	\includegraphics[scale=0.5]{img/f3-10.jpg}
\end{figure}

打开C++ -> General -> Additional Include Directories,点击Additional Include Directories的<Edit...>下拉选项。

\begin{figure}[H]
	\centering
	\includegraphics[scale=0.5]{img/f3-11.jpg}
\end{figure}

添加Vulkan,GLFW和GLM的头文件目录:

\begin{figure}[H]
	\centering
	\includegraphics[scale=0.5]{img/f3-12.jpg}
\end{figure}

接着,打开Linker -> General:

\begin{figure}[H]
	\centering
	\includegraphics[scale=0.5]{img/f3-13.jpg}
\end{figure}

添加Vulkan和GLFW的库目录:

\begin{figure}[H]
	\centering
	\includegraphics[scale=0.5]{img/f3-14.jpg}
\end{figure}

打开 Linker -> Input,点击Additional Dependencies的<Edit...>下拉选项:

\begin{figure}[H]
	\centering
	\includegraphics[scale=0.5]{img/f3-15.jpg}
\end{figure}

添加Vulkan和GLFW的库文件:

\begin{figure}[H]
	\centering
	\includegraphics[scale=0.5]{img/f3-16.jpg}
\end{figure}

现在可以关闭项目属性对话框了。如果一切顺利,我们的代码编辑器里已经没有任何高亮出的错误代码了。

最后,确认我们的代码在64位模式下编译:

\begin{figure}[H]
	\centering
	\includegraphics[scale=0.5]{img/f3-17.jpg}
\end{figure}

然后,按下F5编译运行,你就会看到下面的窗口:

\begin{figure}[H]
	\centering
	\includegraphics[scale=0.5]{img/f3-18.jpg}
\end{figure}

控制台窗口显示的扩展数应该是非0的。至此,我们就配置好了Vulkan的开发环境!

\subsection{Linux}

对于Linux平台的配置,我们使用Ubuntu作为演示,其它Linux平台的配置方法应该是类似的。我们使用二进制包来安装Vulkan SDK,使用GCC 4.8以上版本作为编译器,使用CMake和make作为构建系统。

\subsubsection{Vulkan SDK}

Vulkan SDK是使用Vulkan开发应用程序必不可少的组件。它包含了Vulkan API的头文件,一个校验层实现,调试工具和Vulkan函数加载器。Vulkan函数加载器类似OpenGL的GLEW可以在运行时查询驱动程序支持的Vulkan API函数。

Vulkan SDK可以从LunarG的网站上免费下载。

\begin{figure}[H]
	\centering
	\includegraphics[scale=0.5]{img/f3-19.jpg}
\end{figure}

打开终端,调整当前目录到我们下载的Vulkan SDK安装文件所在目录,然后使用下面的代码运行它:

\begin{lstlisting}[language={bash}]
chmod +x vulkansdk-linux-x86_64-xxx.run
./vulkansdk-linux-x86_64-xxx.run
\end{lstlisting}

安装文件运行后会将Vulkan SDK的所有文件导出到当前目录的VulkanSDK文件夹中。我们可以自己将VulkanSDK文件夹移动到合适的位置。

Vulkan SDK的根目录有一个build\_examples.sh脚本,执行它构建Vulkan SDK的示例程序需要我们安装XCB库,以及一些X窗口的开发文件,可以通过在终端运行下面的代码来安装这些所需的库:

\begin{lstlisting}[language={bash}]
sudo apt install libxcb1-dev xorg-dev
\end{lstlisting}

然后,我们就可以执行build\_examples.sh了:

\begin{lstlisting}[language={bash}]
./build_examples.sh
\end{lstlisting}

如果编译成功,在./examples/build/下就会出现一个cube可执行文件,运行它,可以看到下面的画面:

\begin{figure}[H]
	\centering
	\includegraphics[scale=0.5]{img/f3-20.jpg}
\end{figure}

如果没有看到,而是出现了一条错误消息,可以尝试更新显卡的驱动程序到最新版本,再次尝试,如果仍然出现错误消息,可以在显卡官网查询自己的显卡是否支持Vukan。

\subsubsection{GLFW}

之前提到,Vulkan是一个平台无关的图形API,它没有包含任何用于创建窗口的功能。为了跨平台和避免陷入X11的窗口细节中去,我们使用GLFW库来完成窗口相关操作,GLFW库支持Windows,Linux和MacOS。当然,还有其它一些库可以完成类似功能,比如SDL。但除了窗口相关处理,GLFW库对于Vulkan的使用还有其它一些优点。

这里,由于Vulkan需要较新版本的GLFW才能支持。所以,我们使用源代码来编译安装GLFW。读者可以从GLFW的官方网站免费下载到GLFW的最新源码包。下载完成后,我们将源码包解压,使用终端进入解压的源码所在的文件夹,执行下面的代码生成makefile文件,然后编译GLFW:

\begin{lstlisting}[language={bash}]
cmake .
make
\end{lstlisting}

可能会出现Could NOT find Vulkan的警告信息,可以放心地忽略掉它。编译完成后,使用下面的代码将GLFW安装到系统的库目录中:

\begin{lstlisting}[language={bash}]
sudo make install
\end{lstlisting}

\subsubsection{GLM}

和DirectX 12不同,Vulkan没有包含线性代数库,我们需要自己找一个。GLM就是一个我们需要的线性代数库,它经常和OpenGL一块使用。

GLM是一个只有头文件的库,我们只需要下载它的最新版,然后将它放在一个合适的位置,就可以通过包含头文件的方式使用它。

这里我们直接在终端使用下面的代码安装它:

\begin{lstlisting}[language={bash}]
sudo apt install libglm-dev
\end{lstlisting}

\subsubsection{配置makefile文件}

现在,我们已经安装完了所有的依赖项,可以开始配置应用程序的makefile,验证安装是否正确。

在一个合适的位置新建一个叫做VulkanTest的文件夹,然后在文件夹里创建包含下面代码的main.cpp源代码文件。

\begin{lstlisting}[language={[ANSI]C}]
#define GLFW_INCLUDE_VULKAN
#include <GLFW/glfw3.h>

#define GLM_FORCE_RADIANS
#define GLM_FORCE_DEPTH_ZERO_TO_ONE
#include <glm/vec4.hpp>
#include <glm/mat4x4.hpp>

#include <iostream>

int main() {
	glfwInit();

	glfwWindowHint(GLFW_CLIENT_API, GLFW_NO_API);
	GLFWwindow* window = glfwCreateWindow(800, 600, "Vulkan
window", nullptr, nullptr);

	uint32_t extensionCount = 0;
	vkEnumerateInstanceExtensionProperties(nullptr,
				&extensionCount, nullptr);

	std::cout << extensionCount << " extensions supported" << std::endl;

	glm::mat4 matrix;
	glm::vec4 vec;
	auto test = matrix * vec;

	while(!glfwWindowShouldClose(window)) {
		glfwPollEvents();
	}

	glfwDestroyWindow(window);

	glfwTerminate();

	return 0;
}
\end{lstlisting}

源代码的内容暂时不需要理解,我们现在只是为了验证我们的依赖是否配置正确,源代码的内容,我们会在后面的章节详细说明。

接着,我们需要编写makefile来编译源代码。这里假设读者具有一定makefile使用经验,知道makefile的变量和规则的用法。如果没有,也可以从本教程中快速学习这些知识。

我们首先定义一些变量来简化makefile编写。VULKAN\_SDKPATH变量存放了Vulkan SDK的x86\_64目录的位置:

\begin{lstlisting}[language={bash}]
VULKAN_SDK_PATH = /home/user/VulkanSDK/x.x.x.x/x86_64
\end{lstlisting}

读者应该替换上面代码的路径为自己Vulkan SDK的实际路径。接着,我们定义CFLAGS变量来指定编译选项:

\begin{lstlisting}[language={bash}]
CFLAGS = -std=c++11 -I$(VULKAN_SDK_PATH)/include
\end{lstlisting}

上面的代码表示使用C++ 11来编译源代码,将Vulkan SDK的包含目录加入编译器的包含目录搜索路径中。

然后,定义LDFLAGS变量来指定链接选项:

\begin{lstlisting}[language={bash}]
LDFLAGS = -L$(VULKAN_SDK_PATH)/lib `pkg-config --static --libs glfw3` -lvulkan
\end{lstlisting}

上面的代码将Vulkan SDK的库路径加入链接器的库搜索路径中,链接了Vulkan SDK的vulkan库,使用pkg-config命令取得了glfw静态链接选项。
现在可以开始定义编译VulkanTest的规则了:

\begin{lstlisting}[language={bash}]
VulkanTest: main.cpp
	g++ $(CFLAGS) -o VulkanTest main.cpp $(LDFLAGS)
\end{lstlisting}

验证规则是否正确,可以将上面的代码保存为Makefile文件,然后使用终端在Makefile文件所在目录执行make命令。如果一切顺利,会生成一个VulkanTest可执行文件。

现在,我们定义另外两个规则,test和clean,前一个规则用于执行生成的可执行文件,后一个规则用于清除生成的可执行文件:

\begin{lstlisting}[language={bash}]
.PHONY: test clean

test: VulkanTest
	./VulkanTest

clean:
	rm -f VulkanTest
\end{lstlisting}

验证规则能否执行后,读者可能会发现make clean工作的非常好,但make test却产生了下面的错误信息:

\begin{lstlisting}[language={bash}]
./VulkanTest: error while loading shared libraries:
libvulkan.so.1: cannot open shared object file: No such file or directory
\end{lstlisting}

这是因为libvulkan.so没有被安装在系统的库目录,无法被VulkanTest加载。我们可以通过LD\_LIBRARY\_PATH环境变量显式指定库目录来解决这个问题:

\begin{lstlisting}[language={bash}]
test: VulkanTest
	LD_LIBRARY_PATH=$(VULKAN_SDK_PATH)/lib ./VulkanTest
\end{lstlisting}

现在make test应该可以成功执行VulkanTest了。

\begin{lstlisting}[language={bash}]
test: VulkanTest
	LD_LIBRARY_PATH=$(VULKAN_SDK_PATH)/lib\
	VK_LAYER_PATH=$(VULKAN_SDK_PATH)/etc/explicit_layer.d\
	./VulkanTest
\end{lstlisting}

至此,我们的Makefile文件已经编写完毕了,它的所有内容如下:

\begin{lstlisting}[language={bash}]
VULKAN_SDK_PATH = /home/user/VulkanSDK/x.x.x.x/x86_64

CFLAGS = -std=c++11 -I$(VULKAN_SDK_PATH)/include
LDFLAGS = -L$(VULKAN_SDK_PATH)/lib `pkg-config --static --libs glfw3` -lvulkan

VulkanTest: main.cpp
	g++ $(CFLAGS) -o VulkanTest main.cpp $(LDFLAGS)

.PHONY: test clean

test: VulkanTest
	LD_LIBRARY_PATH=$(VULKAN_SDK_PATH)/lib\
	VK_LAYER_PATH=$(VULKAN_SDK_PATH)/etc/ex	plicit_layer.d\
	./VulkanTest

clean:
	rm -f VulkanTest
\end{lstlisting}

读者可以将刚刚配置的Makefile文件作为一个模板在以后使用。

现在,让我们花点实践浏览下Vulkan SDK目录,在x86\_64/bin目录下还有一个非常有用的程序:glslangValidator。它可以将GLSL代码编译为字节码。我们会在着色器模块章节,对它进行更为详细地说明。除此之外,Bin目录下还包含了Vulkan函数加载器和校验层的二进制文件,它们的库文件则位于Vulkan SDK的Lib目录下。

Vulkan SDK的Doc目录包含了Vulkan SDK的离线文档和完整的Vulkan规范文档。最后是Vulkan SDK的Include目录,它包含了Vulkan API的头文件。除此之外,还有很多文件和目录,但对于我们的教程来说,并没有直接用到它们,所以就不再一一介绍。

至此,我们已经做好开始Vulkan探险之旅的准备!

\subsection{MacOS}

这里假定读者使用Xcode和Homebrew包管理器。另外还需要我们的MacOS版本在10.11以上,显卡设备支持Metal API。

\subsubsection{Vulkan SDK}

Vulkan SDK是使用Vulkan开发应用程序必不可少的组件。它包含了Vulkan API的头文件,一个校验层实现,调试工具和Vulkan函数加载器。Vulkan函数加载器类似OpenGL的GLEW可以在运行时查询驱动程序支持的Vulkan API函数。

Vulkan SDK可以从LunarG的网站上免费下载。

\begin{figure}[H]
	\centering
	\includegraphics[scale=0.5]{img/f3-21.jpg}
\end{figure}

Vulkan SDK的MacOS版本是通过MoltenVK实现的,并非原生实现,也就是说MoltenVK作为一个中间层将Vulkan API调用转换为Metal调用。这也使得我们可以直接使用Metal的调试功能进行调试。

下载Vulkan SDK后,将它解压到一个合适的位置,在解压后的目录中的Applications,可以找到一些Vulkan SDK的演示程序,运行其中的可执行文件cube,你将会看到下面的窗口:

\begin{figure}[H]
	\centering
	\includegraphics[scale=0.5]{img/f3-22.jpg}
\end{figure}

\subsubsection{GLFW}

之前提到,Vulkan是一个平台无关的图形API,它没有包含任何用于创建窗口的功能。为了跨平台和避免陷入窗口操作相关的细节中去,我们使用GLFW库来完成窗口相关操作,GLFW库支持Windows,Linux和MacOS。当然,还有其它一些库可以完成类似功能,比如SDL。但除了窗口相关处理,GLFW库对于Vulkan的使用还有其它一些优点。

我们使用Homebrew包管理器来安装GLFW库。GLFW的3.2.1稳定版目前还尚未完全支持Vulkan,所以我们使用下面的代码安装glfw3包的最新版本:

\begin{lstlisting}[language={bash}]
brew install glfw3 --HEAD
\end{lstlisting}

\subsubsection{GLM}

Vulkan没有包含线性代数库,我们需要自己找一个。GLM就是一个我们需要的线性代数库,它经常和OpenGL一块使用。

GLM是一个只有头文件的库,我们只需要下载它的最新版,然后将它放在一个合适的位置,就可以通过包含头文件的方式使用它。

这里我们直接在终端使用下面的代码安装它:

\begin{lstlisting}[language={bash}]
brew install glm
\end{lstlisting}

\subsubsection{配置Xcode}

现在所有的依赖项已经安装完毕,我们可以开始配置一个最基本的用于Xcode的Vulkan项目。

启动Xcode,然后新建一个Xcode项目,选择Application > Command Line Tool项目类型:

\begin{figure}[H]
	\centering
	\includegraphics[scale=0.5]{img/f3-23.jpg}
\end{figure}

接着,选择C++作为项目使用的语言:

\begin{figure}[H]
	\centering
	\includegraphics[scale=0.5]{img/f3-24.jpg}
\end{figure}

现在将下面的代码作为项目的main.cpp源文件的内容:

\begin{lstlisting}[language={[ANSI]C}]
#define GLFW_INCLUDE_VULKAN
#include <GLFW/glfw3.h>

#define GLM_FORCE_RADIANS
#define GLM_FORCE_DEPTH_ZERO_TO_ONE
#include <glm/vec4.hpp>
#include <glm/mat4x4.hpp>

#include <iostream>

int main() {
	glfwInit();

	glfwWindowHint(GLFW_CLIENT_API, GLFW_NO_API);
	GLFWwindow* window = glfwCreateWindow(800, 600, "Vulkan
window", nullptr, nullptr);

	uint32_t extensionCount = 0;
	vkEnumerateInstanceExtensionProperties(nullptr,
				&extensionCount, nullptr);

	std::cout << extensionCount << " extensions supported" << std::endl;

	glm::mat4 matrix;
	glm::vec4 vec;
	auto test = matrix * vec;

	while(!glfwWindowShouldClose(window)) {
		glfwPollEvents();
	}

	glfwDestroyWindow(window);

	glfwTerminate();

	return 0;
}
\end{lstlisting}

源代码的内容暂时不需要理解,我们现在只是为了验证我们的依赖是否配置正确,源代码的内容,我们会在后面的章节详细说明。

现在Xcode应该会显示一些诸如库未找到的错误。我们接下来的工作就是解决这些错误。在Project Navigator面板选择我们的项目,然后打开Build Settings标签页,进行下面的操作:

\begin{itemize}
	\item 将/usr/local/include加入Header Search Paths,这是Homebrew安装头文件的路径,我们安装的glm和glfw3的头文件都在该文件夹下,然后将vulkansdk/macOS/include加入Header Search Paths,vulkansdk为我们安装的Vulkan SDK的目录。这样Xcode就可以找到我们使用的库的头文件。
	\item 将/usr/local/lib加入Library Search Paths,这是Homebrew安装库文件的路径,我们安装的glm和glfw3的库文件都在该文件夹下,然后将vulkansdk/macOS/lib加入Library Search Paths,vulkansdk为我们安装的Vulkan SDK的目录。这样Xcode就可以找到我们使用的库文件。
\end{itemize}

设置完成后,看起来像这样(实际内容依赖于我们自己的文件所在的位置):

\begin{figure}[H]
	\centering
	\includegraphics[scale=0.5]{img/f3-25.jpg}
\end{figure}

现在,点击 Build Phases标签页,添加glfw3和vulkan框架。这里,为了简便,我们添加的是动态库(如果想要使用静态库,可以参考这些库的官方文档)。

\begin{itemize}
	\item 对于glfw,打开/usr/local/lib目录,可以找到类似libglfw.3.x.dylib形式的文件(x是库的版本号,依赖于我们使用Homebrew下载安装的glfw的版本)。将这个文件拖拽到Linked Frameworks and Libraries标签页即可。
	\item 对于Vulkan,打开vulkansdk/macOS/lib目录(vulkansdk是我们的Vulkan SDK所在目录),拖拽libvulkan.1.dylib和libvulkan.1.x.xx.dylib文件到Linked Frameworks and Libraries标签页即可。
\end{itemize}

完成上面的操作后,更改Copy Files标签页下的Destination为Frameworks,然后清空Subpath文本框,去掉勾选Copy only when installing,点击+号,将所有三个动态库添加进去。

\begin{figure}[H]
	\centering
	\includegraphics[scale=0.5]{img/f3-26.jpg}
\end{figure}

最后,我们需要配置环境变量。在Xcode的工具栏上通过Product > Scheme > Edit Scheme...打开Arguments标签页,添加下面的环境变量:

\begin{itemize}
	\item VK\_ICD\_FILENAMES = vulkansdk/macOS/etc/vulkan/icd.d/MoltenVK\_icd.json
	\item VK\_LAYER\_PATH = vulkansdk/macOS/etc/vulkan/explicit\_layer.d
\end{itemize}

完成后,看起来像这样:

\begin{figure}[H]
	\centering
	\includegraphics[scale=0.5]{img/f3-27.jpg}
\end{figure}

至此为止,我们完成了全部设置,可以编译运行项目了,效果如下:

\begin{figure}[H]
	\centering
	\includegraphics[scale=0.5]{img/f3-28.jpg}
\end{figure}

程序输出的日志信息中的扩展数应该是非0的
。
现在,我们已经做好开始Vulkan探险之旅的准备!

\newpage
